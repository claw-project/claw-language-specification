\section{Utilities}

\subsection{Remove}
\begin{lstlisting}
!$claw remove

  ! code block
  
[!$claw end remove]
\end{lstlisting}

The \textbf{remove} directive allows the user to discard section of code
during the transformation process. 

\textbf{Options and details}\\
If the directive is directly followed by a structured block (\lstinline!IF! or \lstinline!DO!), the
end directive is not mandatory (see example \ref{remove1}). In any other cases, the end
directive is mandatory.

\textbf{Code example}
\label{remove1}

Original code
\begin{lstlisting}
DO k=1, kend
  DO i=1, iend
    ! loop #1 body here
  END DO

  !$claw remove
  IF (k > 1) THEN
    PRINT*, k
  END IF

  DO i=1, iend
    ! loop #2 body here
  END DO
END DO
\end{lstlisting}

Transformed code
\begin{lstlisting}
DO k=1, kend
  DO i=1, iend
    ! loop #1 body here
  END DO

  DO i=1, iend
    ! loop #2 body here
  END DO
END DO
\end{lstlisting}

More code examples in the appendix. Example with block remove (see example \ref{remove2}).
